\documentclass[tikz,border=10pt]{standalone}
\usepackage{tikz}
\usetikzlibrary{shapes.geometric, arrows.meta, positioning, fit, backgrounds}

\begin{document}

\tikzset{
    model/.style={rectangle, rounded corners, minimum width=3.5cm, minimum height=1.2cm,
                  text centered, draw=black, fill=blue!10, font=\bfseries},
    constraint/.style={rectangle, rounded corners, minimum width=3.5cm, minimum height=0.8cm,
                       text centered, draw=black, fill=green!20, font=\small},
    arrow/.style={-Stealth, thick},
    plus/.style={circle, draw=black, fill=orange!30, font=\Large\bfseries,
                 minimum size=0.8cm}
}

\begin{tikzpicture}[node distance=1.5cm]

% Model 1
\node (model1) [model] {Model 1: Vanilla RNN};
\node (arch1) [constraint, below=0.3cm of model1] {Standard RNN};
\node (act1) [constraint, below=0.1cm of arch1] {tanh activation};

% Plus sign
\node (plus1) [plus, below=0.8cm of act1] {+};

% Added constraints for Model 2
\node (add2a) [constraint, below=0.3cm of plus1, fill=yellow!30] {Time constant $\tau$};
\node (add2b) [constraint, below=0.1cm of add2a, fill=yellow!30] {ReLU activation};
\node (add2c) [constraint, below=0.1cm of add2b, fill=yellow!30] {Recurrent noise};

% Model 2
\node (model2) [model, below=0.8cm of add2c] {Model 2: Leaky RNN};

% Plus sign
\node (plus2) [plus, below=0.5cm of model2] {+};

% Added constraints for Model 3
\node (add3) [constraint, below=0.3cm of plus2, fill=yellow!30] {Feedback Alignment};

% Model 3
\node (model3) [model, below=0.8cm of add3] {Model 3: Leaky + FA};

% Plus sign
\node (plus3) [plus, below=0.5cm of model3] {+};

% Added constraints for Model 4
\node (add4a) [constraint, below=0.3cm of plus3, fill=yellow!30] {Dale's Principle (E/I)};
\node (add4b) [constraint, below=0.1cm of add4a, fill=yellow!30] {Sparse Connectivity (20\%)};
\node (add4c) [constraint, below=0.1cm of add4b, fill=yellow!30] {L2 Regularization};

% Model 4
\node (model4) [model, below=0.8cm of add4c] {Model 4: Bio-Realistic};

% Arrows
\draw [arrow] (model1) -- (plus1);
\draw [arrow] (plus1) -- (model2);
\draw [arrow] (model2) -- (plus2);
\draw [arrow] (plus2) -- (model3);
\draw [arrow] (model3) -- (plus3);
\draw [arrow] (plus3) -- (model4);

% Labels on the side
\node[anchor=west, font=\footnotesize, text width=3cm] at (7, -2.5) {
    \textbf{Baseline:} \\
    No biological \\
    constraints
};

\node[anchor=west, font=\footnotesize, text width=3cm] at (7, -7) {
    \textbf{Temporal:} \\
    Brain-like \\
    time constants
};

\node[anchor=west, font=\footnotesize, text width=3cm] at (7, -11) {
    \textbf{Learning:} \\
    Biologically \\
    plausible gradients
};

\node[anchor=west, font=\footnotesize, text width=3cm] at (7, -16.5) {
    \textbf{Architecture:} \\
    Full biological \\
    realism
};

\end{tikzpicture}

\end{document}
